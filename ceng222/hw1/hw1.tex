\documentclass[12pt]{article}
\usepackage[utf8]{inputenc}
\usepackage{float}
\usepackage{amsmath} 
\usepackage{graphicx}
\usepackage{listings}

\lstset{
  language=Octave,
  basicstyle=\ttfamily,
  keywordstyle=\bfseries,
  commentstyle=\itshape,
  columns=fullflexible,
  keepspaces=true,
  showstringspaces=false,
  breaklines=true
}
\usepackage[hmargin=3cm,vmargin=6.0cm]{geometry}
%\topmargin=0cm
\topmargin=-2cm
\addtolength{\textheight}{6.5cm}
\addtolength{\textwidth}{2.0cm}
%\setlength{\leftmargin}{-5cm}
\setlength{\oddsidemargin}{0.0cm}
\setlength{\evensidemargin}{0.0cm}

%misc libraries goes here

\begin{document}

\section*{Student Information } 
%Write your full name and id number between the colon and newline
%Put one empty space character after colon and before newline
Full Name : Tahsin ELMAS     \\
Id Number :  2476844 \\

% Write your answers below the section tags
\section*{Answer 1}

\subsection*{a)} 
To find out the expected value of rolling one die for each color, we can use the following method:\\
$$\mathbf{E}(X)=\sum_x x P(x)$$
For the blue die, there are six face values ranging from 1 to 6, with an equivalent probability as follows:  $$P\{1\}=P\{2\}=P\{3\}=P\{4\}=P\{5\}=P\{6\}=\frac{1}{6}$$ 
As a result, the expected value of rolling one die is:

\begin{align*}
E\{\text { blue }\}&=(1\times \frac{1}{6})+ (2\times \frac{1}{6})+(3\times \frac{1}{6})+(4\times \frac{1}{6})+(5\times \frac{1}{6})+(6\times \frac{1}{6})\\
&=(\frac{1}{6})+(\frac{2}{6})+(\frac{3}{6})+(\frac{4}{6})+(\frac{5}{6})+(\frac{6}{6})\\
&= \frac{21}{6}\\
&= 3.5
\end{align*}

For the yellow die, the face values range  $1$ ,$3$ ,$4$ and $8$  with probabilities as follows: \\
\begin{center}
$P\{1\}=\frac{3}{8}$ and $P\{3\} = \frac{3}{8}$ and $P\{4\} = \frac{1}{8}$ and $P\{8\} = \frac{1}{8}$
\end{center}

Therefore, the expected value of a single  roll is:
\begin{align*}
E\{\text { yellow }\} &= (1 \times \frac{3}{8}) +(3 \times \frac{3}{8})+(4 \times \frac{1}{8})+(8 \times \frac{1}{8})\\
&= \frac{3}{8}+\frac{9}{8}+\frac{4}{8}+\frac{8}{8}\\
&= \frac{24}{8}\\
&= 3
\end{align*}
\newpage
For the red die, the face values range  $2$, $3$, $4$ and $6$ with probabilities as follows:\\
\begin{center}
$P\{2\} = \frac{5}{10}$ and $P\{3\} = \frac{2}{10}$ and $P\{4\} = \frac{2}{10}$ and $P\{6\} = \frac{1}{10}$
\end{center}

Therefore, the expected value of a single roll is:
\begin{align*}
E\{\text { red }\}&=(2 \times \frac{5}{10})+(3 \times \frac{2}{10})+(4 \times \frac{2}{10})+(6 \times \frac{1}{10})\\
&=(\frac{10}{10})+(\frac{6}{10})+(\frac{8}{10})+(\frac{6}{10})\\
&=\frac{30}{10}\\
&=3
\end{align*}
\subsection*{b)} 
If we roll three blue dice, the resulting sum could range from 3 to 18, with varying probabilities for each value. On the other hand, if we roll one die of each color, we may obtain a greater range of values and a higher possible sum. To determine the optimal choice, we can compute the expected value for each scenario. Specifically, when rolling three blue dice, the expected value for the total is:\\
$$E\{\text { 3  blue }\} ={3}\times E\{\text { blue }\} =3 \times 3.5 =10.5$$

For rolling one die of each color, the expected value of the total is:
\begin{center}
$E\{\text { blue + yellow + red }\}&=E\{\text { blue }\}+E\{\text { yellow }\}+E\{\text {red}\}&=3.5+3+3 &=9.5$
\end{center}

Therefore, we would prefer to roll three blue dice to maximize the total value.

\subsection*{c)} 
If it is guaranteed that the yellow die's value will be 8 , then the expected value of rolling one die of each color changes to:
\begin{center}
$E\{\text { blue, red , yellow = 8 }\}=E\{\text { blue }\}+E\{\text { red }\} + 8 &=3.5+ 3 + 8 =14.5$
\end{center}

For rolling three blue dice, the expected value remains the same at $\mathrm{E}(3$ blue $)&=10.5$ as part (a). Therefore, in this case, we would prefer to roll one die of each color to maximize the total value.

\subsection*{d)} 
To find the probability that the rolled die is red, given that the value of the die is $3$ , we can use Bayes' theorem:

$$P\{\text  { Red } \mid \text { Value }= 3\}= \frac{P\{\text { Value }=3 \mid \text { Red }\} \times P\{\text { Red} \}}{P\{\text { Value }=3\}} $$

We know that the probability of rolling a $3$ on a red die is $0.2$,
and We also know that each color has an equal probability of being chosen: 

$$P(\text { Red }) =P(\text { Blue }) =P(\text { Yellow }) = \frac{1}{3} = 0.33$$

So the probability of rolling a $3$  on any die is:
\begin{align*}
P\{\text {Value} =3\} &= P\{3 \mid \text {Blue}\} \times P\{\text {Blue}\}+ P\{3 \mid \text {Yellow }\} \times P\{\text {Yellow}\} + P\{ 3 \mid \text {Red}\}\times P\{\text {Red}\} \\
&=( \frac{1}{6} \times \frac{1}{3}) + ( \frac{3}{8} \times \frac{1}{3})+( \frac{2}{10} \times \frac{1}{3})\\
&= 0.2472
\end{align*}


Therefore:
\begin{align*}
P\{\text { Red } \mid \text { Value }=3\} &= \frac{0.2 \times 0.33}{0.2472}\\
&= 0.2669
\end{align*}

So the probability that the rolled die is red, given that the value of the is $3$ , is $0.2669$ .

\subsection*{e)} 
To find the probability that the total value will be $5$ when a single blue die and a single yellow die is rolled together, we need to calculate the total number of possible outcomes and the number of outcomes that result in a total value of $5$ .\\
The possible outcomes for rolling a blue die and a yellow die are as follows:\\
\begin{center}
Blue die: $\{1,2,3,4,5,6\}$\\
Yellow die: $\{1,1,1,3,3,3,4,8\}$\\  
\end{center}

To calculate the total number of possible outcomes, we multiply the number of outcomes for the blue die (6) by the number of outcomes for the yellow die (8), which gives us $6 \times 8 =48$ possible outcomes.

To calculate the number of outcomes that result in a total value of $5$ , we need to consider all possible combinations of numbers from the blue and yellow dice that add up to $5$. These combinations are:
\begin{center}
$P\{5 \mid Blue \cap Yellow\} = P\{\text{Blue} = 1\} \times P\{ \text{Yellow}= 4\}$\\
$P\{5 \mid Blue \cap Yellow\} =P\{\text{Blue} = 2\} \times P\{ \text{Yellow}= 3\}$\\
$P\{5 \mid Blue \cap Yellow\} =P\{\text{Blue} = 4\} \times P\{ \text{Yellow}= 1\}$\\

\end{center}


The probability of each of these combinations can be calculated by multiplying the probability of rolling the specific number on the blue die by the probability of rolling the specific number on the yellow die.\\
\begin{center}
$$P\{\text{Blue} = 1\} \times P\{ \text{Yellow}= 4\} = (\frac{1}{6}) \times(\frac{1}{8})= \frac{1}{48}$$
$$P\{\text{Blue} = 2\} \times P\{ \text{Yellow}= 3\}=(\frac{1}{6}) \times (\frac{3}{8})=\frac{3}{48}$$
$$P\{\text{Blue} = 4\} \times P\{ \text{Yellow}= 1\}=(\frac{1}{6}) \times (\frac{3}{8})=\frac{3}{48}$$
\end{center}


The total probability of getting a total value of 5 is the sum of the probabilities of all the possible combinations:\\
\begin{center}
$$P\{\text{Value } = 5\} = (P\{\text{Blue} = 1\} \times P\{ \text{Yellow}= 4\} )+\\ (P\{\text{Blue} = 2\} \times P\{ \text{Yellow}= 3\}) + (P\{\text{Blue} = 4\} \times P\{ \text{Yellow}= 1\}) $$\\
$$P\{\text{ Value } = 5\} = (\frac{1}{48})+(\frac{3}{48})+(\frac{3}{48})=\frac{7}{48} $$
$$P\{\text{ Value } = 5\} = 0.1458 $$
\end{center}

Therefore, the probability that the total value will be $5$ when a single blue die and a single yellow die is rolled together is $0.1458$.

\section*{Answer 2}
\subsection*{a)} 
We can use the binomial distribution to determine the probability that at least four distributors from Company A will offer a discount tomorrow. We can define X as the number of distributors out of the 80 who offer a discount on that day, which follows a binomial distribution with a probability of success of $p=0.025$ and a sample size of $X=80$. We are interested in finding the probability of X being greater than or equal to $P\{X \ge 4\}$.\\
To find the probability $P\{X \ge 4\} = 1 -  F(3)$, we need to first calculate the cumulative distribution function (cdf) of the binomial distribution with parameters n = 80 and p = 0.025. The cdf $F(3)$ gives the probability that X is less than or equal to $3$, where X is the number of distributors out of the 80 who offer a discount on a specific day.
$$
F(x)=\boldsymbol{P}\{X \leq x\}=\sum_{k=0}^x\left(\begin{array}{c}
n \\
k
\end{array}\right) p^k(1-p)^{n-k}
$$
To calculate F(3) using the formula for the binomial CDF, we need to plug in the values of $n=80$, $p=0.025$ , and $q= 1- p = 0.975$, and evaluate the sum from k=0 to 3.
$$
F(3)=\boldsymbol{P}\{X \leq 3\}=\sum_{k=0}^x\left(\begin{array}{c}
80 \\
3
\end{array}\right) 0.025^k(0.975)^{n-k} = 0.8584
$$

\begin{center}
$P\{X \ge 4\} = 1 -  F(3) = 1 - 0.8594 = 0.1406 $\\
\end{center}

\subsection*{b)} 
Let $A$ and $B$ denote the events that a phone can be bought from company A and B, respectively. Let $P\{A\}$ and $P\{B\}$ denote the probabilities of these events, respectively. Then, the probability that a phone can be bought in two days is given by:

$$
P(A \cup B)=P(A)+P(B)-P(A \cap B)
$$
Since the two companies act independently of each other, we have $P(A \cap B)=P(A) \times P(B)$. Therefore,
$$
P(A \cup B)=P(A)+P(B)-P(A) \times P(B)
$$
To calculate $P(A)$ and $P(B)$, we can use the complement rule. Let $\bar{A}$ and $\bar{B}$ denote the events that a phone cannot be bought from company $A$ and $B$, respectively. Then, we have $P(A)=1-P(\bar{A})$ and $P(B)=1-P(\bar{B})$. Since company $A$ has $80$ distributors and each of them offers a discount with probability 0.025 , the probability that a phone can be bought from company A on a specific day is:
$$
\begin{aligned}
P(A)=1-(1-0.025)^{80} = 0.8680 \\
\end{aligned}
$$
Since company B has a single distributor and it offers a discount with probability 0.1 , the probability that a phone can be bought from company B on a specific day is:
$$
P(B)=0.1
$$
Therefore, the probability that a phone can be bought in two days is:
$$
\begin{aligned}
& P(A \cup B)= 0.8680 + 0.1- (0.8680  \times 0.1) = 0.8812
\end{aligned}
$$
Thus, the probability that you can buy a phone in two days is approximately 0.8812 . We used the binomial distribution for part (a) and the complement rule and the multiplication rule for part (b).
\newpage
\section*{Answer 3}
Here's  code simulates a dice rolling game with three colored dice, comparing two different ways of playing the game. It rolls the dice 1000 times, keeping track of the total value of each option and how often option 2 is better. It calculates the average total value and the percentage of times option 2 is better, then prints the results to the console. Code is below and you can see the  output of the octave online in APPENDIX part at the end of the paper.
\begin{lstlisting}
blue_die = [1 2 3 4 5 6];
yellow_die = [1 1 1 3 3 3 4 8];
red_die = [2 2 2 2 2 3 3 4 4 6];

n_iterations = 1000;

total_value_option1 = 0;
total_value_option2 = 0;

count_option2_better = 0;

for i = 1:n_iterations
    blue_roll = blue_die(randi([1 6]));
    yellow_roll = yellow_die(randi([1 8]));
    red_roll = red_die(randi([1 10]));
    total_value_option1 = total_value_option1 + blue_roll + yellow_roll + red_roll;
    
    blue_rolls = randi([1 6], [1 3]);
    total_value_option2 = total_value_option2 + sum(blue_rolls);

    if sum(blue_rolls) > (blue_roll + yellow_roll + red_roll)
        count_option2_better = count_option2_better + 1;
    end
end
average_total_value_option1 = total_value_option1 / n_iterations;
average_total_value_option2 = total_value_option2 / n_iterations;
percentage_option2_better = (count_option2_better / n_iterations) * 100;
fprintf("Average total value for option 1: %f\n", average_total_value_option1);
fprintf("Average total value for option 2: %f\n", average_total_value_option2);
fprintf("Percentage of times option 2 is better: %f%%\n", percentage_option2_better);
\end{lstlisting}
\newpage
\section*{APPENDIX}

Here's the screenshot of the code and output from octave online:

\begin{figure}[h]
\centering
\includegraphics[scale=0.4]{output.png}
\caption{Q3 -The code and output from octave online.}
\label{fig:image}
\end{figure}



\end{document}